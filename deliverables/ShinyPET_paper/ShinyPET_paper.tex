\documentclass{acm_proc_article-sp}
\usepackage[utf8]{inputenc}

\renewcommand{\paragraph}[1]{\vskip 6pt\noindent\textbf{#1 }}
\usepackage{hyperref}
\usepackage{graphicx}
\usepackage{url}

\providecommand{\tightlist}{%
  \setlength{\itemsep}{0pt}\setlength{\parskip}{0pt}}

\title{ShinyPET: A Predictive, Exploratory and Text RShiny Application
using Airbnb data}


% Add imagehandling

\numberofauthors{2}
\author{
\alignauthor Ang Su Yiin \\
        \affaddr{Singapore Management University}\\
       \email{\href{mailto:suyiin.ang.2020@mitb.smu.edu.sg}{\nolinkurl{suyiin.ang.2020@mitb.smu.edu.sg}}}
\and \alignauthor Joey Chua \\
        \affaddr{Singapore Management University}\\
       \email{\href{mailto:joey.chua.2020@mitb.smu.edu.sg}{\nolinkurl{joey.chua.2020@mitb.smu.edu.sg}}}
\and \alignauthor Kevin Gunawan Albindo \\
        \affaddr{Singapore Management University}\\
       \email{\href{mailto:kgalbindo.2019@mitb.smu.edu.sg}{\nolinkurl{kgalbindo.2019@mitb.smu.edu.sg}}}
\and }

\date{}

%Remove copyright shit
\permission{}
\conferenceinfo{} {}
\CopyrightYear{}
\crdata{}

% Pandoc syntax highlighting

% Pandoc citation processing
\newlength{\csllabelwidth}
\setlength{\csllabelwidth}{3em}
\newlength{\cslhangindent}
\setlength{\cslhangindent}{1.5em}
% for Pandoc 2.8 to 2.10.1
\newenvironment{cslreferences}%
  {}%
  {\par}
% For Pandoc 2.11+
\newenvironment{CSLReferences}[3] % #1 hanging-ident, #2 entry spacing
 {% don't indent paragraphs
  \setlength{\parindent}{0pt}
  % turn on hanging indent if param 1 is 1
  \ifodd #1 \everypar{\setlength{\hangindent}{\cslhangindent}}\ignorespaces\fi
  % set entry spacing
  \ifnum #2 > 0
  \setlength{\parskip}{#2\baselineskip}
  \fi
 }%
 {}
\usepackage{calc} % for calculating minipage widths
\newcommand{\CSLBlock}[1]{#1\hfill\break}
\newcommand{\CSLLeftMargin}[1]{\parbox[t]{\csllabelwidth}{#1}}
\newcommand{\CSLRightInline}[1]{\parbox[t]{\linewidth - \csllabelwidth}{#1}}
\newcommand{\CSLIndent}[1]{\hspace{\cslhangindent}#1}

\usepackage{graphicx}
\usepackage{float}
\usepackage{caption}
\captionsetup{skip=1pt}
\setlength{\textfloatsep}{2pt}
\setlength{\intextsep}{2pt}

\begin{document}
\maketitle

\begin{abstract}
The increasing availability of data has resulted in the increased demand
for data driven decisions. Although there is an extensive range of
commercial statistical tools, they are often subscription-based and
demand good technical knowledge to mine and draw insights from.
Therefore, it may not appeal to the average user. Using a collection of
R packages available, ShinyPET, an R-Shiny application is developed for
the average user to perform exploratory and confirmatory analysis, text
mining and predictive analysis, as well as to formulate insights and
make data-driven decisions. Airbnb data provides a baseline for this
application as the data generated is rich in information, consisting of
structured, unstructured, and location data. This paper discusses the
design framework, use case and future works of the ShinyPET dashboard.
\end{abstract}

\emph{Keywords} - Airbnb, Exploratory Analysis, Confirmatory Analysis,
Text Mining, Predictive Analytics, Decision Making, R Shiny, Interactive
Data Visualisation.

\hypertarget{introduction}{%
\section{Introduction}\label{introduction}}

With increasing affordable data storage and processing technologies, the
demand for data-driven decision-making (DDDM)\footnote{DDDM refers to
  the systematic analysis, examination and integration of data to making
  strategic decisions, rather than based on intuition or observation
  alone (Mandinach,
  2012){[}\protect\hyperlink{ref-doi:10.1080ux2f00461520.2012.667064}{3}{]}}
has increased significantly. As Geoffrey Moore opines, ``Without big
data analytics, companies are blind and deaf, wandering out onto the Web
like deer on a freeway.'' With the use of data driven decision making
through analytics tools, firms performance would improve (Yasmin, M et
al.,
2020){[}\protect\hyperlink{ref-https:ux2fux2fdoi.orgux2f10.1016ux2fj.jbusres.2020.03.028}{5}{]}

Airbnb is an online vacation rental maravailableketplace servicing a
community of hosts and travellers. By 2020, Airbnb has millions of
listings in over 220 counties and regions across 100,000 cities
{[}\protect\hyperlink{ref-Airbnb2020}{\textbf{Airbnb2020?}}{]}. The data
generated provides rich information, including structured data
e.g.~price and location, as well as unstructured data e.g.~reviews and
listing descriptions. Thus, Airbnb provides a good use case and base
case for exploratory and confirmatory analysis, text mining, and
predictive modeling as presented in our ShinyPET dashboard.

\hypertarget{motivation-of-the-application}{%
\section{Motivation of the
application}\label{motivation-of-the-application}}

The motivation of this project stems from two main issues - the
proliferation of data and lack of user-friendly tools to make
data-driven decision. According to Harris
(2012){[}\protect\hyperlink{ref-https:ux2fux2fhbr.orgux2f2012ux2f09ux2fdata-is-useless-without-the-skills}{\textbf{https://hbr.org/2012/09/data-is-useless-without-the-skills?}}{]},
data is impractical without the ability to analyse it. Although there is
a wide range of commercial statistics and analytics tools , these tools
are often subscription-based and require technical knowledge to mine and
draw insights from. On the other hand, while open source tools as such
Python and R allow for data visualisations, users would require
extensive programming background to generate such insights.

Hence, this project aims to develop an interface which is concise,
interactive, and user-friendly using R Shiny. With this interface,
data-based decisions can be made from the interactive GUI. The R Shiny
App will cover 3 modules: 1) Exploratory - users are able to draw
interesting patterns based on selected variables, which are augmented by
statistical tests based on the chosen variables. 2) Text - users are
able to perform analysis on textual data such as reviews to generate
quantitative insights.\\
3) Predictive - users are able to prepare and build a variety of
prediction models without the need to have in-depth understanding of
predictive models and their algorithms.\\
This application can be extended to Airbnb data from other countries,
and also to other datasets.

\hypertarget{review-and-critic-on-past-works}{%
\section{Review and critic on past
works}\label{review-and-critic-on-past-works}}

Radiant application {[}\protect\hyperlink{ref-radiant2019}{4}{]}, an
open-source platform-independent browser-based interface for business
analytics in R, illustrates the robustness of Rshiny for web-based
application. Developed to promote quick and reproducible data analytics,
the application provides interactivity and flexibility in performing
visualisation, statistical and predictive analysis. However, there are
limitations to the application. First, in terms of exploratory data
analysis, most of the plots produced are of static nature which can be
enhanced by wrapping plotly around them. Secon, for statistical testing,
users are expected to have a basic understanding of statistical testing
methods as they are first required to select their testing method, which
is further enhanced by automating testing methods based on inputs. In
addition, newer packages such as visNetwork can be applied for
interactive tree visualisation that in turn improves the assessment of
decision tree model. Last, in terms of visualisation, the statistical
testing and charts are placed in separate tabs.A single-page view would
enhance the aesthetics and usability.

Text Mining with R
Book{[}\protect\hyperlink{ref-https:ux2fux2fwww.tidytextmining.comux2findex.html}{\textbf{https://www.tidytextmining.com/index.html?}}{]}
authored by Silge and Robinson presents a comprehensive approach to
handle text. First, the book is content-heavy, which may not be
appealing to the typical users. Second, tidytext used for data wrangling
and visusaliastion is widely used thus allow users to apply such methods
easily. However, these tools require technical skills from users and are
not interactive. To allow easy usage and enhance interactivity, packages
such as plotly and highchart can be used. Highcharter has various themes
and features such as tooltips which greatly enahnce visualisation.

Tidymodels {[}\protect\hyperlink{ref-tidymodels2020}{1}{]} has gained
interest by providing a framework for predictive modeling and machine
learning. It is aligned with the tidyverse principles which leads to a
tidier and consistent grammar in the predictive analytics process.
Different models offered in Radiant package are also available for
implementation in Tidymodels framework, which is why our application
leverages Tidymodel as the main framework to conduct predictive
analytics on Airbnb data.

Lu, Y., Garcia, R., Hansen, B. et al.~(2017)
{[}\protect\hyperlink{ref-https:ux2fux2fdoi.orgux2f10.1111ux2fcgf.13210}{2}{]}
provides a comprehensive summary of research on Predictive Visual
Analytics. The paper discusses how visual analytics systems are
implemented to support predictive analytics process such as feature
selection, incremental learning, model comparison and result
exploration. The overall goal of visual analytics is to support
explanation in each step of predictive analytics exercise which is also
our motivation in developing this application.

\hypertarget{design-framework}{%
\section{Design framework}\label{design-framework}}

The design of our shinyPET is based on ensuring a comprehensive data
analysis coupled with aesthetics. Taking into account the user point of
view, 3 main principles, namely user-friendliness, interactivity and
ease of understanding are adopted.

To get started, the introduction page provides an overview of the
application. This allow users to have an understanding of the case that
he/she will be exploring.

In the exploratory page, data summary and their tabular form would be
presented for user's understanding of data. In the explore sub tab,
users are able to visualise provided data based on various variables.
This user-friendliness and interactivity provides flexibility and ease
of use without needing any technical knowledge.

In the text page, various text mining technique tools are presented in
visualiastion formats such as wordclouds and barcharts. Concepts such as
sentiment analysis are simplified and recerated into visualisations for
ease of understanding.

In the predictive page, \emph{to add on}

Aesthetically, the application's colour scheme should be based on the
theme of the topic. Using our case of Airbnb, the official colours are
Raush, Babu and Foggy (type of gray).

The combination of the 3 principles are consistently incorporated into
various steps of the data analysis in the 3 modules, hence providing
users an easy and comprehensive way to make data-driven decisions.

\hypertarget{exploratory-module}{%
\subsection{Exploratory module}\label{exploratory-module}}

The exploratory module enables users to perform exploratory and
confirmatory analysis on selected variables to identify interesting
patterns. There are three sections in this module - observe, map and
confirm \& explore.

\hypertarget{observe-submodule}{%
\subsubsection{Observe submodule}\label{observe-submodule}}

As shown in Figure {[}1{]}, the Observe section provides a summary of
the data for users to quickly understand and form questions surrounding
the data. Hence, this section was designed mainly based on ease of
understanding principles.

\begin{figure}[H]

{\centering \includegraphics[width=1\linewidth]{images/design_observe} 

}

\caption{Interface and components of Observe section}\label{fig:unnamed-chunk-1}
\end{figure}

There are two main components - first is the top 4 boxes that provide an
overview of the data - number of variables, observations and data type.
The second component is the tables below shows the summary of each
variable by respective data type. The tables allow for some
interactivity - search boxes allows user to filter the data accordingly,
while the arrow icons next to the variables names allow users to sort
the data according to their needs.

\hypertarget{map-submodule}{%
\subsubsection{Map submodule}\label{map-submodule}}

\begin{figure}[H]

{\centering \includegraphics[width=1\linewidth]{images/design_map} 

}

\caption{Interface and components of Map section}\label{fig:unnamed-chunk-2}
\end{figure}

The Map section, Figure {[}2{]}, the Map section allows user to explore
the geographic patterns of Airbnb listings through thematic maps. Thus,
this section was designed based on the three principles stated above and
partially based on Shneiderman's interactive dynamics principle of
``overview, zoom and filter, then details on demand,'' save for the
`zoom and filter' portion as it was not applicable to this data.

As such, there are 2 main components of this submodule - the map which
provides a macro overview of the Airbnb listings by the selected
variable. The second component is the table, which provides details of
the map.

\hypertarget{map-submodule-1}{%
\subsubsection{Map submodule}\label{map-submodule-1}}

The Explore and Confirm section, figure {[}3{]}, enables user to explore
and perform inferential statistics based on their exploration and
questions generated from the previous two sections.

\begin{figure}[H]

{\centering \includegraphics[width=1\linewidth]{images/design_explore1} 

}

\caption{Interface and components of Explore and Confirm section}\label{fig:unnamed-chunk-3}
\end{figure}

There are 3 main components - the selection input on the left, the
statistical results and the chart.

The selection input was designed to be interactive and user-friendly,
allowing users to customise charts based on the drop-down list provided.
The application provides for 4 types of chart namely: distribution,
mosaic, boxplot and scatter plot. In drop down menus will change
according to the selected chart type, for example, if the `Distribution'
chart was selected, only the x-variable drop-down input will be shown.

\begin{figure}[H]

{\centering \includegraphics[width=1\linewidth]{images/design_explore2} 

}

\caption{Graph's manipulation function of the Explore and Confirm section}\label{fig:unnamed-chunk-4}
\end{figure}

The chart was designed according to Shneiderman's interactive dynamics
of highlight, filter or manipulate. This graph allows for users to
manipulate views by selecting a single object in a plot, highlighting
selected records and defining a region on the graph. Furthermore, the
plotted chart can be downloaded for users to communicate their findings.
See figure {[}4{]} for examples

Given that the application is tailored towards users that are not well
versed in statistics, the statistical test was designed to be easy to
understand, thus the test methods and results are automated based on the
selected variables. An interactive slider is provided for user to easily
adjust statistical test.

\hypertarget{text-module}{%
\subsection{Text module}\label{text-module}}

The text module usitlise various text mining techniques to transform
unstructured text i.e.~reviews into structured format to identify
patterns and bring about meaninigful insights.

Prior to application of text mining techniques, text preprocessing has
to be carried out. This involves the use of tokenisation, stemming and
lematisation. Tokenisation is the process of splitting a column of
reviews into tokens such that they are flatenned into the format of
one-token-per-row. Stemming is the process of seperating the prefixes
and suffixes from words to derive the root word form and meaning.
Stemming algorithms work by cutting off the end or the beginning of the
word, taking into account a list of common prefixes and suffixes that
can be found in an inflected word. However, the stemming method changes
words such as earlier to earli and checking to checkin as shown above.
As such, this process was excluded.Lemmatization, on the other hand,
takes into consideration the morphological analysis of the words.

\hypertarget{token-frequency-submodule}{%
\subsubsection{Token Frequency
Submodule}\label{token-frequency-submodule}}

To visualise token frequency, wordcloud is commonly used. Worldcloud
provides an easy way to show how frequent a word appears in a corpus. In
wordcloud, the size of a word indicates how frequent the word appears in
a given text.

Other than condisdering words as individual units, ``ngrams'' are also
used to tokenise pairs of adajacent words. ngrams provide context in
sentiment analysis. For instance, while the word ``happy'' can be
positive, in a sentence which containts the words ``not happy'' would
mean otherwise. Hence, performing sentiment analysis on bigram allow us
to examine sentiment-associated words.

\begin{figure}[H]

{\centering \includegraphics[width=1\linewidth]{images/tokenfrequency} 

}

\caption{Interface and components of Explore and Confirm section}\label{fig:unnamed-chunk-5}
\end{figure}

There are two components: on the left is the wordcloud, and on the right
is the bar chart the ranks the frequency of word in descending order.
From the chart, it can be observed that the words ``clean,'' ``stay,''
``location,'' and ``nice'' occurred most frequently. This could suggest
that cleanliness of the environement/room is mentioned the most; it
could mean that travellers generally prefer clean room. Additionally,
location is mentioned the second-most, suggesting that a location nearer
to city for convenience, or q quiet location is important as customers.

\hypertarget{sentiment-analysis-submodule}{%
\subsubsection{Sentiment Analysis
Submodule}\label{sentiment-analysis-submodule}}

In this submodeule, 3 dictionaries were used to plot wordcloud that
shows both the frequency and sentiments. First, AFINN lexicon measurs
sentiment with a numeric score between -5 to 5. BING categorises words
as either positive or negative. NRC categorise words into emotions.

\begin{figure}[H]

{\centering \includegraphics[width=1\linewidth]{images/sentimentanalysis} 

}

\caption{Interface and components of Explore and Confirm section}\label{fig:unnamed-chunk-6}
\end{figure}

Users can select the various lexicons to view the wordcloud. To further
value add to the wordcloud, a further visualisation is conducted. For
AFINN, bar chart is plotted to show the spread and weightage of
sentiments. For BING, the distribution of negative, positive and netural
sentiments are shown. For NRC, a radial plot to show the tendency for
customers to lean towards is shown. From the chart, it shows that
sentiments generally lean towards positive and joy, while less on
sadness and surprise.

\hypertarget{topic-modelling-submodule}{%
\subsubsection{Topic Modelling
Submodule}\label{topic-modelling-submodule}}

Latent Dirichlet allocation is an example of topic modeling algorithm,
based on 2 principles: 1. Every document is a mixture of topics.For
example, document A is 90\% topic on location and 10\% on host's
hospitality. Whereas, document B is 30\% topic on location and 70\% on
host's hospitality. 2. Every topic is a mixture of words. For instance,
based on Airbnb data, one topic can be cleanliness, and the other topic
can on amenities.

\begin{figure}[H]

{\centering \includegraphics[width=1\linewidth]{images/topicmodelling} 

}

\caption{Interface and components of Explore and Confirm section}\label{fig:unnamed-chunk-7}
\end{figure}

Users are able to select the number of topics they would like to
understand further. For isntance, by selecting 2 topics, we can observe
that the most common word in topic 1 is ``x,'' ``y,'' ``z.'' THe most
common words in topic 2 include ``a,'' ``b,'' ``c.'' We can also observe
that there are overalapping topics. An example of usage in topic
modelling would be to improve search algorithe,/suggestions for usesr.
By anyalysing the topics and develop subtopics, the most relevant
listings e.g.~cleanest room, or closest location could be shown to the
users.

\hypertarget{correlation-network-submodule}{%
\subsubsection{Correlation Network
Submodule}\label{correlation-network-submodule}}

Word occurences and correlations are commonly used to identify family of
words.

\begin{figure}[H]

{\centering \includegraphics[width=1\linewidth]{images/correlationnetwork} 

}

\caption{Interface and components of Explore and Confirm section}\label{fig:unnamed-chunk-8}
\end{figure}

There are 2 options. First the is X graph. Second is the X graph. From
the diagram, is can be seen that thereis strong connection between words
such as ``X'' and ``X,'' ``Y'' and ``Y'' and ``Y.''

\hypertarget{predictive-module}{%
\subsection{Predictive module}\label{predictive-module}}

Our predictive module design framework follows Tidymodels framework for
data pre-processing, model training, tuning, and validation. On top of
that, feature selection are supported by other R packages such as
ggcorplot (for correlation matrix), ranger and Boruta (for feature
importance). The visualisation and interactivity are embedded in each
step of predictive analytics as explained below.

Data sampling - Selection of training-test split proportion provides
flexibility in deciding how to spend data budget on the model
development process. The distribution plot between training and test set
highlights any potential bias in the training data set.

\begin{figure}[H]

{\centering \includegraphics[width=1\linewidth]{images/datasplit} 

}

\caption{Data sampling and distribution plot}\label{fig:unnamed-chunk-9}
\end{figure}

Feature selection - Correlation matrix with customised correlation type
and p-value criteria, as well as variable importance allow assessment of
correlation among variables.

\begin{figure}[H]

{\centering \includegraphics[width=1\linewidth]{images/featselect} 

}

\caption{Correlation matrix and variable importance}\label{fig:unnamed-chunk-10}
\end{figure}

Data transformation - Transformation steps from recipe package and plot
between pre and post processing step increases user awareness on what
transformation steps are performed and on which variables.

\begin{figure}[H]

{\centering \includegraphics[width=1\linewidth]{images/recipetrf} 

}

\caption{Data transformation steps}\label{fig:unnamed-chunk-11}
\end{figure}

Model training - Coefficient estimate or decision tree information as
interactive plot to improve result evaluation.

\begin{figure}[H]

{\centering \includegraphics[width=1\linewidth]{images/mdltrn} 

}

\caption{Training result evaluation}\label{fig:unnamed-chunk-12}
\end{figure}

Model validation - Rsquare plot to visualise validation result along
with table of metric performance.

\begin{figure}[H]

{\centering \includegraphics[width=1\linewidth]{images/mdleval} 

}

\caption{Validation result evaluation}\label{fig:unnamed-chunk-13}
\end{figure}

Prediction error assessment - Training set distribution plot is
overlapped with predicted values to allow further assessment on
prediction error.

\begin{figure}[H]

{\centering \includegraphics[width=1\linewidth]{images/prederror} 

}

\caption{Prediction error assessment}\label{fig:unnamed-chunk-14}
\end{figure}

Hyper-parameter tuning - Plot of model performance using different
hyper-parameters setting helps user to understand the change in
performance.

\begin{figure}[H]

{\centering \includegraphics[width=1\linewidth]{images/hypartune} 

}

\caption{Hyper-parameter tuning result}\label{fig:unnamed-chunk-15}
\end{figure}

Model selection - Plot of performance metrics from different models to
support model selection process.

\begin{figure}[H]

{\centering \includegraphics[width=1\linewidth]{images/mdlcompare} 

}

\caption{Models performance comparison}\label{fig:unnamed-chunk-16}
\end{figure}

The combination of these three modules along with its interactivity and
usability would empower users to make data driven decisions with the
insights generated.

\hypertarget{case-study-airbnb-singapore}{%
\section{Case Study : Airbnb
Singapore}\label{case-study-airbnb-singapore}}

We selected Airbnbs in Singapore data as our case study due to its
variety of data - structured, textual and location data.

Since founded in 2008, Airbnb has revolutionised the tourism industry
and has become one of the largest sharing economy. In 2019, Airbnb
generated USD4.7b in global revenue with close to 7 million listings on
its
website{[}\protect\hyperlink{ref-airbnb2021}{\textbf{airbnb2021?}}{]}.
Airbnb plays an important role of connecting hosts and travelers who
would not typically interact in absence of the platform. However, given
the sheer number of Airbnb listings available, Airbnb hosts face stiff
competition not only from other Airbnb hosts, but also from hotels and
service apartments. Meanwhile, for Airbnb guests, going through each
listings and their reviews on Airbnb website can be quite overwhelming
and a tedious process.

Hence, this application allows users to analyse their needs and compare
across other listings.

The two main data used for analysis are the listings and reviews of
Singapore Airbnb data, which were obtained from InsideAirbnb on the 27
January 2021. The listing dataset contains close to four thousand
listings with 74 variables, while the review dataset consist of fifty
thousand reviews and 6 variables.

In order to reduce the loading time of the application, the datasets
were preprocess prior with only the cleaned dataset loaded. In
additional to the usual pre-processing steps, redundant variables such
as listing id, url id, were removed. Additionally, unstructured
variables were converted into structured variables by counting the
length of the text.

\hypertarget{geographical-distribution-of-airbnbs}{%
\subsection{Geographical distribution of
Airbnbs}\label{geographical-distribution-of-airbnbs}}

\begin{figure}[H]

{\centering \includegraphics[width=1\linewidth]{images/usecase_explore} 

}

\caption{Point symbol map on the left, choropleth on the right}\label{fig:unnamed-chunk-17}
\end{figure}

The point symbol map reveals that Airbnbs are distributed throughout
Singapore, with high concentration around town center. The 4 distinct
hotspots are (1) Geylang/Kallang, (2) Lavender/ Rochor / Bugis , (3)
Orchard and (4) Chinatown. Area (3) and (4) are mainly tourist areas -
(3) Orchard is the main shopping belt of Singapore, while (4) Chinatown
retains significant historical and cultural landmarks. Areas (1) and (2)
are popular due to its low price per person (see choropleth of figure x)
while staying relatively close to town. Additionally, we noticed that
Airbnbs tend to be located along the MRT station track. Given that
Airbnb guests tend to be more budget conscious, they would most likely
use public transportation.

Through the use of these maps, potential Airbnb investor can identify
areas that are highly saturated and the average price per person of that
area to estimate their investment yield before committing to the
investment.

\hypertarget{distribution-of-review-score-rating}{%
\subsection{Distribution of review score
rating}\label{distribution-of-review-score-rating}}

\begin{figure}[H]

{\centering \includegraphics[width=1\linewidth]{images/usecase_explore2} 

}

\caption{Distribution of review score rating}\label{fig:unnamed-chunk-18}
\end{figure}

The overall Airbnb listing review score - `review\_scores\_rating,' is
capped at 100 and has a left skewed distribution. This seems to suggest
that either most guests tend to score their reviews positively, or
Airbnbs with low ratings tend to delist and exit the Airbnb market.

\hypertarget{confirmatory-analysis-of-superhost-and-review-scores-hosting}{%
\subsection{Confirmatory analysis of superhost and review scores
hosting}\label{confirmatory-analysis-of-superhost-and-review-scores-hosting}}

\begin{figure}[H]

{\centering \includegraphics[width=1\linewidth]{images/usecase_explore4} 

}

\caption{Exploratory and Confirmatory Analysis on superhost status and review scores}\label{fig:unnamed-chunk-19}
\end{figure}

Figure x suggests that listings with the superhost status tend to have
higher review scores as indicated the statistical test results where
p-value is less than alpha of 0.05.

\hypertarget{observing-correlation-among-variables}{%
\subsection{Observing correlation among
variables}\label{observing-correlation-among-variables}}

Data sets like Airbnb are rich with large numbers of variable. However,
multicolinearity among variables are known to affect predictive model
performance. Correlation matrix helps us to avoid such case by
highlighting variables with high correlation value. In our example
below, we observe correlations within rating score components, listing
availability period, and review components. With this information, we
can then select our variables more wisely.

\begin{figure}[H]

{\centering \includegraphics[width=1\linewidth]{images/corrcase} 

}

\caption{Correlation among variables}\label{fig:unnamed-chunk-20}
\end{figure}

\hypertarget{model-explanation}{%
\subsection{Model explanation}\label{model-explanation}}

In predicting listing price using linear model, the plot of coefficient
estimate helps to explain the trained model. In the example below, our
interface allows sorting of variables based on p-value score where
variables with lowest p-value is located on top. Property type which
falls under ``Others'' category (those with counts of less than 5\% in
the data set) has the lowest p-value score and positive estimate, which
may represent unique property type (e.g.~boat, campsite, chalet, villa)
where the listing price is above the average price of common property
type like apartment and condominium (as shown in the boxplot from
exploratory module). Amenities and beds are also in the top 5 predictor
where it correlates positively with listing price. However, the error
bar is wider for property type ``Others'' as compared to the amenities
and beds, representing more uncertainty in the estimate value.

\begin{figure}[H]

{\centering \includegraphics[width=1\linewidth]{images/LMcoeff} 

}

\caption{Coefficient estimate and boxplot from exploratory module}\label{fig:unnamed-chunk-21}
\end{figure}

\hypertarget{discussion}{%
\section{Discussion}\label{discussion}}

What has the audience learned from your work? What new insights or
practices has your system enabled? A full blown user study is not
expected, but informal observations of use that help evaluate your
system are encouraged.

\hypertarget{future-work}{%
\section{Future Work}\label{future-work}}

Shiny PET was built with Singapore's Airbnb dataset as a usecase for
using R Shiny to perform exploratory and confirmatory, text and
predictive analytics without users needing extensive programming or
statistical knowledge. Hence, the application could be further enhance
by including a data load and wrangling function to accommodate different
datasets.

Additionally, the current types of chart and statistical test are
limited with only 4 types of charts and parametric statistical test for
each chart type respectively. Other charts, such as violin and bar
charts, can be incorporated further. Additional hypothesis testing
methods can be included such as non-parametric test for median,
statistical test by pairs and others. The current application only
supports two types of map, other spatial maps such as kernel density map
and navigation map can be included. Moreover, the explore and text
module can be combined with views coordinated and linked to provide
multiple dimensional exploration.

The current predictive module is limited to 5 types of predictive model.
In future, more predictive models can be added to the list, such as
neural network to provide user with wider model selection. In terms of
hyper-parameter tuning, parameters can be made available for user input
to provide more flexibility in developing predictive model. In-depth
statistical analysis in model training such as residual analysis are
currently not available and this would be a good additional tool to
improve our application.

\hypertarget{acknowledgement}{%
\section{Acknowledgement}\label{acknowledgement}}

The authors wish to thank Professor Kam Tin Seong of Singapore
Management University for his extensive guidance and support during this
project.

\hypertarget{references}{%
\section*{References}\label{references}}
\addcontentsline{toc}{section}{References}

\hypertarget{refs}{}
\begin{CSLReferences}{0}{0}
\leavevmode\hypertarget{ref-tidymodels2020}{}%
\CSLLeftMargin{{[}1{]} }
\CSLRightInline{Kuhn, M. and Wickham, H. 2020. \emph{Tidymodels: A
collection of packages for modeling and machine learning using tidyverse
principles.}}

\leavevmode\hypertarget{ref-https:ux2fux2fdoi.orgux2f10.1111ux2fcgf.13210}{}%
\CSLLeftMargin{{[}2{]} }
\CSLRightInline{Lu, Y. et al. 2017. The state-of-the-art in predictive
visual analytics. \emph{Computer Graphics Forum}. 36, 3 (2017),
539--562.}

\leavevmode\hypertarget{ref-doi:10.1080ux2f00461520.2012.667064}{}%
\CSLLeftMargin{{[}3{]} }
\CSLRightInline{Mandinach, E.B. 2012. A perfect time for data use: Using
data-driven decision making to inform practice. \emph{Educational
Psychologist}. 47, 2 (2012), 71--85.}

\leavevmode\hypertarget{ref-radiant2019}{}%
\CSLLeftMargin{{[}4{]} }
\CSLRightInline{Nijs, V. 2019. \emph{Radiant -- business analytics using
r and shiny}.}

\leavevmode\hypertarget{ref-https:ux2fux2fdoi.orgux2f10.1016ux2fj.jbusres.2020.03.028}{}%
\CSLLeftMargin{{[}5{]} }
\CSLRightInline{Yasmin, M. et al. 2020. Big data analytics capabilities
and firm performance: An integrated MCDM approach. \emph{Business
Research}. 114, (2020), 1--15.}

\end{CSLReferences}
\setlength{\parindent}{0in}

\end{document}
